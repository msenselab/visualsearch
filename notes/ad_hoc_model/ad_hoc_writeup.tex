\documentclass[12pt]{article}
\usepackage{color,graphicx,amsmath,authblk}
\usepackage[a4paper]{geometry}

% MACROS %%%%%%%%%%%%%%%%%%%%%%%%%%%%%%%%%%%%%%%%%%%%%%%%%%%%

\newcommand{\mus}{\mu_\text{S}}

%%%%%%%%%%%%%%%%%%%%%%%%%%%%%%%%%%%%%%%%%%%%%%%%%%%%%%%%%%%%%%

\begin{document}
\title{A simplified model for the observing and responding to visual stimuli}
\author{Berk Gercek, Luigi Acerbi, Reidar Riveland}
\maketitle
\section{Task Structure}
We consider a visual search task in which subjects are presented displays according to one of two conditions. A \emph{dynamic} condition in which a set of L-shaped distractor stimuli are rearranged randomly every 100 ms, and possibly contain a T-shaped \emph{target}. All stimuli can be rotated in 90$^\circ$ increments. In the \emph{static} condition, the positions of the stimuli are fixed for the duration of the trial.
In both conditions, the target is present with probability $\frac{1}{2}$.

The subjects must respond regarding the presence or absence of a target in order to receive a reward for correctly identifying the state. In this model we will only address the dynamic task.

\section{Generative Model}
The underlying state of the display can be one of two states: Target Present ($S=1$) or Target Absent ($S=0$). From this state $S$, we model the observer's measurements of the display as i.i.d. noisy observations $x_t$ drawn from a Gaussian distribution at each time step $t$.
\begin{align}
\begin{split}
S &\in \{0, 1\} \\
P(x_t | S) &= \mathcal{N}(x_t | \mus, \sigma^2)\\
\end{split}
\end{align}
where we assume the same measurement variance $\sigma^2$ for the target and all distractors in the display (this assumption might be alleviated later).

From which we can obtain the posterior
\begin{align}
P(S | x_{1 \dots t}) \propto P(S) \prod_{t^\prime=1}^t \mathcal{N}(x_{t^\prime} | \mus, \sigma^2)
\end{align}
and in the following we assume that observer knows that $P(S) = \frac{1}{2}$.

\section{Decision Values}
We suppose that the observer must at each time step $t$ decide whether or not to respond regarding the state of the display. This decision is influenced by the reward given for each possible pair of decisions $i$ and underlying states $j$, which we define as $R_{ij}$. We can formalize the values surrounding this decision as

\begin{align}
V_t = \max
\begin{cases}
\text{wait}, & \langle V_{t+1} \rangle -\rho\\
r=1, & P(S=1 | x_{1 \dots t})R_{11} + P(S=0|x_{1 \dots t})R_{10}\\
r=0, & P(S=0 | x_{1 \dots t})R_{00} + P(S=1 | x_{1 \dots t})R_{01}\\
\end{cases}
\end{align}

In which $\rho$ is the reward rate, and $\langle V_{t+1} \rangle$ is the reward expected by observer at the next time step. 

The solution of equation above remains to be done.
\end{document}
