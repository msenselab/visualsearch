\documentclass[12pt]{article}
\usepackage{color,graphicx,amsmath,authblk, gensymb}
\usepackage[a4paper]{geometry}

\begin{document}
\title{A simplified model for the observing and responding to visual stimuli}
\author{Berk Gercek, Luigi Acerbi, Reidar Riveland}
\maketitle
\section{Task Structure}
We are given a behavioral task in which subjects are presented with one of two stimuli; A dynamic stimuli in which a field of L-shaped distractors are rearranged every 100ms, and potentially contain a T-shaped target (All stimuli can be rotated in 90\degree \ increments) or a static task in which the positions of the distractors and target are fixed.

The targets must respond regarding the presence or absence of a target in order to receive a reward for correctly identifying the state. In this model we will only address the dynamic task.

\section{Generative Model}
We presume that the underlying state of the system can be one of two states: Target Present or Target Absent. From this state $S$ we model the observer's samples from the system as noisy observations at each time step $x_t$ drawn from a gaussian.

\begin{align*}
S &\in {0, 1} \\
P(x_t | S) &= \mathcal{N}(x_t | \mu_s, \sigma^2)\\
\end{align*} 

From which we can obtain a MAP estimate for the state of the system

\begin{align}
P(S | x_{1 \dots t}) = \prod_1^t \mathcal{N}(x_t | \mu_s, \sigma^2)
\end{align}

\section{Decision Values}
We suppose that the observer must at each time step $t$ decide whether or not to respond regarding the state of the system. This decision is influenced by the reward given for each possible pair of decision $i$ and underlying state $j$, which we define as $R_{ij}$. We can formalize the values surrounding this decision as

\begin{align}
V_t = max
\begin{cases}
wait, & \langle V_{t+1} \rangle -\rho\\
r=1, & P(S=1 | x_{1 \dots t})R_{11} + P(S=0|x_{1 \dots t})R_{10}\\
r=0, & P(S=0 | x_{1 \dots t})R_{00} + P(S=1 | x_{1 \dots t})R_{01}\\
\end{cases}
\end{align}

In which $\rho$ is the reward rate, and $\langle V_{t+1} \rangle$ is the expected value of the system at the next time step.  
\end{document}
